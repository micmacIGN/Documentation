\documentclass{beamer}

\usepackage[utf8]{inputenc}
\usepackage[T1]{fontenc}

\title{There Is No Largest Prime Number}
\subtitle{Beamer demonstration: longer subtitle}
\institute{IGN}
\date[ISPN ’80]{27th International Symposium of Prime Numbers}
\author[Euclid]{Euclid of Alexandria \texttt{euclid@alexandria.edu}}

\usetheme{ign}

\begin{document}

    \begin{frame}[plain]
        \titlepage{}
    \end{frame}

    \section{Introduction}
        \begin{frame}{Itroduction}
            Volupti
        \end{frame}
        \subsection{subsection name}
            \begin{frame}{There Is No Largest Prime Number}
                \framesubtitle{The proof uses \textit{reductio ad absurdum}.}
                \begin{theorem}
                    There is no largest prime number. \end{theorem}
                \begin{enumerate}
                    \item<1-| alert@1> Suppose $p$ were the largest prime number.
                    \item<2-> Let $q$ be the product of the first $p$ numbers.
                    \item<3-> Then $q+1$ is not divisible by any of them.
                    \item<1-> But $q + 1$ is greater than $1$, thus divisible by some prime
                    number not in the first $p$ numbers.
                \end{enumerate}
            \end{frame}
        \subsection{subsection 2}
        \begin{frame}
            Text qui volupti
        \end{frame}
\end{document}
